%iffalse
\let\negmedspace\undefined
\let\negthickspace\undefined
\documentclass[journal,12pt,twocolumn]{IEEEtran}
\usepackage{cite}
\usepackage{amsmath,amssymb,amsfonts,amsthm}
\usepackage{algorithmic}
\usepackage{graphicx}
\usepackage{textcomp}
\usepackage{xcolor}
\usepackage{txfonts}
\usepackage{listings}
\usepackage{enumitem}
\usepackage{mathtools}
\usepackage{gensymb}
\usepackage{comment}
\usepackage[breaklinks=true]{hyperref}
\usepackage{tkz-euclide} 
\usepackage{listings}
\usepackage{gvv}
\usepackage{multicol}
%\def\inputGnumericTable{}                                 
\usepackage[latin1]{inputenc}                                
\usepackage{color}                                            
\usepackage{array}                                            
\usepackage{longtable}                                       
\usepackage{calc}                                             
\usepackage{multirow}                                         
\usepackage{hhline}                                           
\usepackage{ifthen}                                           
\usepackage{lscape}
\usepackage{tabularx}
\usepackage{array}
\usepackage{float}


\newtheorem{theorem}{Theorem}[section]
\newtheorem{problem}{Problem}
\newtheorem{proposition}{Proposition}[section]
\newtheorem{lemma}{Lemma}[section]
\newtheorem{corollary}[theorem]{Corollary}
\newtheorem{example}{Example}[section]
\newtheorem{definition}[problem]{Definition}
\newcommand{\BEQA}{\begin{eqnarray}}
\newcommand{\EEQA}{\end{eqnarray}}
\newcommand{\define}{\stackrel{\triangle}{=}}
\theoremstyle{remark}
\newtheorem{rem}{Remark}

% Marks the beginning of the document
\begin{document}
\bibliographystyle{IEEEtran}
\vspace{3cm}

\title{10/A/C/5-19}
\author{EE24BTECH11040 - Mandara Hosur}
\maketitle
\newpage
\bigskip

\renewcommand{\thefigure}{\theenumi}
\renewcommand{\thetable}{\theenumi}

\section*{\textbf{\textcolor{magenta}{C. MCQs with One Correct Answer}}}

\textbf{5.} If $f(x) = cos(ln x)$, then $$f(x)f(y)-\frac{1}{2} \left[f\left(\frac{x}{y}\right)+f(xy)\right]$$ has the value

\hfill{\textcolor{magenta}{\textbf{(1983 - 1 Mark)}}}

\begin{multicols}{2}
	\begin{enumerate}
		\item[(a)] -1 
		\item[(b)] $\frac{1}{2}$
		\item[(c)] -2 
		\item[(d)] none of these
	\end{enumerate}
\end{multicols}

\textbf{6.} The domain of definition of the function
$$y = \frac{1}{\log_{10}{(1-x)}} + \sqrt{x+2}$$ is

\hfill{\textcolor{magenta}{\textbf{(1983 - 1 Mark)}}}

\begin{multicols}{2}
	\begin{enumerate}
		\item[(a)] (-3, -2) excluding -2.5 
		\item[(b)] [0, 1] excluding 0.5
		\item[(c)] [-2, 1) excluding 0 
		\item[(d)] none of these
	\end{enumerate}
\end{multicols}

\textbf{7.} Which of the following functions is periodic?

\hfill{\textcolor{magenta}{\textbf{(1983 - 1 Mark)}}}

\begin{enumerate}
\item[(a)] $f(x)=x-\left[x\right]$ where $\left[x\right]$ denotes the largest integer less than or equal to the real number $x$
\item[(b)] $f(x)=sin\frac{1}{x}$ for $x\neq0$, $f(0)=0$
\item[(c)] $f(x)=xcosx$
\item[(d)] none of these
\end{enumerate}

\textbf{8.} Let $f(x)=sinx$ and $g(x)=ln|x|$. If the ranges of the composition functions $fog$ and $gof$ are $R_1$ and $R_2$ respectively, then 

\hfill{\textcolor{magenta}{\textbf{(1994 - 2 Marks)}}}

\begin{enumerate}
\item[(a)] $R_1=\{u:-1\le u<1\}$, $R_2=\{v:-\infty<v<0\}$
\item[(b)] $R_1=\{u:-\infty<u<0\}$, $R_2=\{v:-1\le v\le0\}$
\item[(c)] $R_1=\{u:-1<u<1\}$, $R_2=\{v:-\infty<v<0\}$
\item[(d)] $R_1=\{u:-1\le u\le1\}$, $R_2=\{v:-\infty<v\le0\}$
\end{enumerate}

\textbf{9.} Let $f(x)=(x+1)^{2}-1$, $x\ge-1$. Then the set $\{x:f(x)=f^{-1}(x)\}$ is

\hfill{\textcolor{magenta}{\textbf{(1995)}}}

\begin{enumerate}
\item[(a)] \{0, -1, $\frac{-3+i\sqrt{3}}{2}$, $\frac{-3-i\sqrt{3}}{2}$\}
\item[(b)] \{0, 1, -1\}
\item[(c)] \{0, -1\}
\item[(d)] empty
\end{enumerate}

\textbf{10.} The function $f(x)=|px-q|+r|x|$, $x\in(-\infty,\infty)$ where $p>0$, $q>0$, $r>0$ assumes its minimum value only on one point if

\hfill{\textcolor{magenta}{\textbf{(1995)}}}

\begin{multicols}{2}
	\begin{enumerate}
		\item[(a)] $p\neq q$
		\item[(b)] $r\neq q$
		\item[(c)] $r\neq p$ 
		\item[(d)] $p=q=r$
	\end{enumerate}
\end{multicols}

\textbf{11.} Let $f(x)$ be defined for all $x>0$ and be continuous. Let $f(x)$ satisfy $f\left(\frac{x}{y}\right)=f(x)-f(y)$ for all $x$, $y$ and $f(e)=1$. Then

\hfill{\textcolor{magenta}{\textbf{(1995S)}}}

\begin{multicols}{2}
	\begin{enumerate}
		\item[(a)] $f(x)$ is bounded 
		\item[(b)] $f\left(\frac{1}{x}\right)\to0$ as $x\to0$
		\item[(c)] $xf(x)\to1$ as $x\to0$ 
		\item[(d)] $f(x)=lnx$
	\end{enumerate}
\end{multicols}

\textbf{12.} If the function $f:[1,\infty)\to[1,\infty)$ is defined by $f(x)=2^{x(x-1)}$, then $f^{-1}(x)$ is

\hfill{\textcolor{magenta}{\textbf{(1999 - 2 Marks)}}}

\begin{multicols}{2}
	\begin{enumerate}
		\item[(a)] $\left(\frac{1}{2}\right)^{x(x-1)}$ 
		\item[(b)] $\frac{1}{2}\left(1+\sqrt{1+4log_{2}x}\right)$
		\item[(c)] $\frac{1}{2}\left(1-\sqrt{1+4log_{2}x}\right)$ 
		\item[(d)] not defined
	\end{enumerate}
\end{multicols}

\textbf{13.} Let $f:R\to R$ be any function. Define $g:R\to R$ by $g(x)=|f(x)|$ for all $x$. Then $g$ is

\hfill{\textcolor{magenta}{\textbf{(2000S)}}}

\begin{enumerate}
\item[(a)] onto if $f$ is onto
\item[(b)] one-one if $f$ is one-one
\item[(c)] continuous if $f$ is continuous
\item[(d)] differentiable if $f$ is differentiable
\end{enumerate}

\textbf{14.} The domain of definition of the function $f(x)$ given by the equation $2^{x}+2^{y}=2$ is

\hfill{\textcolor{magenta}{\textbf{(2000S)}}}

\begin{multicols}{2}
	\begin{enumerate}
		\item[(a)] $0<x\le1$ 
		\item[(b)] $0\le x\le1$
		\item[(c)] $-\infty<x\le0$ 
		\item[(d)] $-\infty<x<1$
	\end{enumerate}
\end{multicols}

\textbf{15.} Let $g(x)=1+x-[x]$ and
\begin{equation}
f(x)=
\begin{cases}
-1, & \text{$x<0$} \\
0, & \text{$x=0$.} \\
1, & \text{$x>0$}
\end{cases}
\end{equation}
Then for all $x$, $f(g(x))$ is equal to

\hfill{\textcolor{magenta}{\textbf{(2001S)}}}

\begin{multicols}{2}
	\begin{enumerate}
		\item[(a)] $x$ 
		\item[(b)] 1
		\item[(c)] $f(x)$ 
		\item[(d)] $g(x)$
	\end{enumerate}
\end{multicols}

\textbf{16.} If $f:[1,\infty)\to[2,\infty)$ is given by $f(x)=x+\frac{1}{x}$ then $f^{-1}(x)$ equals

\hfill{\textcolor{magenta}{\textbf{(2001S)}}}

\begin{multicols}{2}
	\begin{enumerate}
		\item[(a)] $\frac{(x+\sqrt{x^{2}-4})}{2}$ 
		\item[(b)] $\frac{x}{(1+x^{2})}$
		\item[(c)] $\frac{(x-\sqrt{x^{2}-4})}{2}$ 
		\item[(d)] $1+\sqrt{x^{2}-4}$
	\end{enumerate}
\end{multicols}

\textbf{17.} The domain of definition of $f(x)=\frac{log_{2}{(x+3)}}{x^{2}+3x+2}$ is

\hfill{\textcolor{magenta}{\textbf{(2001S)}}}

\begin{multicols}{2}
	\begin{enumerate}
		\item[(a)] $R \backslash \{-1,-2\}$ 
		\item[(b)] $(-2,\infty)$
		\item[(c)] $R \backslash \{-1,-2,-3\}$ 
		\item[(d)] $(-3,\infty)\backslash\{-1,-2\}$
	\end{enumerate}
\end{multicols}

\textbf{18.} Let $E=\{1,2,3,4\}$ and $F=\{1,2\}$. Then the number of onto functions from E to F is

\hfill{\textcolor{magenta}{\textbf{(2001S)}}}

\begin{multicols}{4}
	\begin{enumerate}
		\item[(a)] 14 
		\item[(b)] 16 
		\item[(c)] 12 
		\item[(d)] 8
	\end{enumerate}
\end{multicols}

\textbf{19.} Let $f(x)=\frac{\alpha x}{x+1}$, $x\neq-1$. Then, for what value of $\alpha$ is $f(f(x))=x$?

\hfill{\textcolor{magenta}{\textbf{(2001S)}}}

\begin{multicols}{4}
	\begin{enumerate}
		\item[(a)] $\sqrt{2}$ 
		\item(b) $-\sqrt{2}$ 
		\item[(c)] 1 
		\item[(d)] -1
	\end{enumerate}
\end{multicols}

\end{document}
