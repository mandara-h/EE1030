%iffalse
\let\negmedspace\undefined
\let\negthickspace\undefined
\documentclass[journal,12pt,twocolumn]{IEEEtran}
\usepackage{cite}
\usepackage{amsmath,amssymb,amsfonts,amsthm}
\usepackage{algorithmic}
\usepackage{graphicx}
\usepackage{textcomp}
\usepackage{xcolor}
\usepackage{txfonts}
\usepackage{listings}
\usepackage{enumitem}
\usepackage{mathtools}
\usepackage{gensymb}
\usepackage{comment}
\usepackage[breaklinks=true]{hyperref}
\usepackage{tkz-euclide} 
\usepackage{listings}
\usepackage{gvv}
\usepackage{multicol}
%\def\inputGnumericTable{}                                 
\usepackage[latin1]{inputenc}                                
\usepackage{color}                                            
\usepackage{array}                                            
\usepackage{longtable}                                       
\usepackage{calc}                                             
\usepackage{multirow}                                         
\usepackage{hhline}                                           
\usepackage{ifthen}                                           
\usepackage{lscape}
\usepackage{tabularx}
\usepackage{array}
\usepackage{float}


\newtheorem{theorem}{Theorem}[section]
\newtheorem{problem}{Problem}
\newtheorem{proposition}{Proposition}[section]
\newtheorem{lemma}{Lemma}[section]
\newtheorem{corollary}[theorem]{Corollary}
\newtheorem{example}{Example}[section]
\newtheorem{definition}[problem]{Definition}
\newcommand{\BEQA}{\begin{eqnarray}}
\newcommand{\EEQA}{\end{eqnarray}}
\newcommand{\define}{\stackrel{\triangle}{=}}
\theoremstyle{remark}
\newtheorem{rem}{Remark}

% Marks the beginning of the document
\begin{document}
\bibliographystyle{IEEEtran}
\vspace{3cm}

\title{13/A/E/6-21}
\author{EE24BTECH11040 - Mandara Hosur}
\maketitle
\newpage
\bigskip

\renewcommand{\thefigure}{\theenumi}
\renewcommand{\thetable}{\theenumi}

\section*{\textbf{E. Subjective Problems}}

\begin{enumerate}

\item 
\begin{enumerate}
\item $PQ$ is a vertical tower. $P$ is the foot and $Q$ is the top of the tower. $A$, $B$, $C$ are three points in the horizontal plane through P. The angles of elevation of $Q$ from $A$, $B$, $C$ are equal, and each is equal to $\theta$. The sides of the triangle $ABC$ are $a$, $b$, $c$; and the area of the triangle $ABC$ is $\Delta$. Show that the height of the tower is $\frac{abc\tan{\theta}}{4\Delta}$.
\item $AB$ is a vertical pole. The end $A$ is on the level ground. $C$ is the middle point of $AB$. $P$ is a point on the level ground. The portion $CB$ subtends an angle $\beta$ at $P$. If $AP=nAB$ then show that $\tan{\beta}=\frac{n}{2n^2+1}$.
\end{enumerate}

\hfill{\brak{1980}}

\item Let the angles $A$, $B$, $C$ of a triangle $ABC$ be in A.P. and let $b:c=\sqrt{3}:\sqrt{2}$. Find the angle $A$. 

\hfill{\brak{1981 - 2 Marks}}

\item A vertical pole stands at a point $Q$ on a horizontal ground. $A$ and $B$ are points on the ground, $d$ meters apart. The pole subtends angles $\alpha$ and $\beta$ at $A$ and $B$ respectively. $AB$ subtends an angle $\gamma$ at $Q$. Find the height of the pole. 

\hfill{\brak{1982 - 3 Marks}}

\item Four ships $A$, $B$, $C$ and $D$ are at sea in the following relative positions: $B$ is on the straight line segment $AC$, $B$ is due North of $D$ and $D$ is due west of $C$. The distance between $B$ and $D$ is 2 km. $\angle{BDA}=40\degree$, $\angle{BCD}=25\degree$. What is the distance between $A$ and $D$? $\sbrak{\text{Take} \sin{25\degree}=0.423}$

\hfill{\brak{1983 - 3 Marks}}

\item The ex-radii $r_1$, $r_2$, $r_3$ of $\Delta ABC$ are in H.P. Show that its sides $a$, $b$, $c$ are in A.P.

\hfill{\brak{1983 - 3 Marks}} 

\item For a triangle $ABC$ it is given that $\cos{A}+\cos{B}+\cos{C}=\frac{3}{2}$. Prove that the triangle is equilateral. 

\hfill{\brak{1984 - 4 Marks}}

\item With usual notation, if in a triangle $ABC$; $\frac{b+c}{11}=\frac{c+a}{12}=\frac{a+b}{13}$ then prove that $\frac{\cos{A}}{7}=\frac{\cos{B}}{19}=\frac{\cos{C}}{25}$. 

\hfill{\brak{1984 - 4 Marks}}

\item A ladder rests against a wall at an angle $\alpha$ to the horizontal. Its foot is pulled away from the wall through a distance $a$, so that it slides a distance $b$ down the wall making an angle $\beta$ with the horizontal. Show that $a=b\tan{\frac{1}{2}\brak{\alpha+\beta}}$.

\hfill{\brak{1985 - 5 Marks}}

\item In a triangle $ABC$, the median to the side $BC$ is of length $\frac{1}{\sqrt{11-6\sqrt{3}}}$ and it divides the angle $A$ into angles $30\degree$ and $45\degree$. Find the length of the side $BC$.

\hfill{\brak{1985 - 5 Marks}}

\item If in a triangle $ABC$, $\cos{A}\cos{B}+\sin{A}\sin{B}\sin{C}=1$, show that $a:b:c=1:1:\sqrt{2}$.

\hfill{\brak{1986 - 5 Marks}}

\item A sign-post in the form of an isosceles triangle $ABC$ is mounted on a pole of height $h$ fixed to the ground. The base $BC$ of the triangle is parallel to the ground. A man standing on the ground at a distance $d$ from the sign-post finds that the top vertex $A$ of the triangle subtends an angle $\beta$ and either of the other two vertices subtends the same angle $\alpha$ at his feet. Find the area of the triangle. 

\hfill{\brak{1988 - 5 Marks}}

\item $ABC$ is a triangular park with $AB=AC=100$m. A television tower stands at the midpoint of $BC$. The angles of elevation of the top of the tower at $A$, $B$, $C$ are $45\degree$, $60\degree$, $60\degree$, respectively. Find the height of the tower. 

\hfill{\brak{1989 - 5 Marks}}

\item A vertical tower $PQ$ stands at a point $P$. Points $A$ and $B$ are located to the South and East of $P$ respectively. $M$ is the mid point of $AB$. $PAM$ is an equilateral triangle; and $N$ is the foot of the perpendicular from $P$ on $AB$. Let $AN=20$ metres and the angle of elevation of the top of the tower at $N$ is $\tan^{-1}{2}$. Determine the height of the tower and the angles of elevation of the top of the tower at $A$ and $B$.

\hfill{\brak{1990 - 4 Marks}}

\item The sides of a triangle are three consecutive natural numbers and its largest angle is twice the smallest one. Determine the sides of the triangle.

\hfill{\brak{1991 - 4 Marks}}

\item In a triangle of base $a$ the ratio of the other two sides is $r\brak{<1}$. Show that the altitude of the triangle is less than or equal to $\frac{ar}{1-r^2}$.

\hfill{\brak{1991 - 4 Marks}}

\item A man notices two objects in a straight line due west. After walking a distance $c$ due north he observes that the objects subtend an angle $\alpha$ at his eye; and, after a further distance $2c$ due north, and angle $\beta$. Show that the distance between the objects is $\frac{8c}{3\cot{\beta}-\cot{\alpha}}$; the height of the man is being ignored. 

\hfill{\brak{1991 - 4 Marks}}

\end{enumerate}

\end{document}
