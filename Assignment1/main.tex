%iffalse
\let\negmedspace\undefined
\let\negthickspace\undefined
\documentclass[journal,12pt,twocolumn]{IEEEtran}
\usepackage{cite}
\usepackage{amsmath,amssymb,amsfonts,amsthm}
\usepackage{algorithmic}
\usepackage{graphicx}
\usepackage{textcomp}
\usepackage{xcolor}
\usepackage{txfonts}
\usepackage{listings}
\usepackage{enumitem}
\usepackage{mathtools}
\usepackage{gensymb}
\usepackage{comment}
\usepackage[breaklinks=true]{hyperref}
\usepackage{tkz-euclide} 
\usepackage{listings}
\usepackage{gvv}
\usepackage{multicol}
%\def\inputGnumericTable{}                                 
\usepackage[latin1]{inputenc}                                
\usepackage{color}                                            
\usepackage{array}                                            
\usepackage{longtable}                                       
\usepackage{calc}                                             
\usepackage{multirow}                                         
\usepackage{hhline}                                           
\usepackage{ifthen}                                           
\usepackage{lscape}
\usepackage{tabularx}
\usepackage{array}
\usepackage{float}


\newtheorem{theorem}{Theorem}[section]
\newtheorem{problem}{Problem}
\newtheorem{proposition}{Proposition}[section]
\newtheorem{lemma}{Lemma}[section]
\newtheorem{corollary}[theorem]{Corollary}
\newtheorem{example}{Example}[section]
\newtheorem{definition}[problem]{Definition}
\newcommand{\BEQA}{\begin{eqnarray}}
\newcommand{\EEQA}{\end{eqnarray}}
\newcommand{\define}{\stackrel{\triangle}{=}}
\theoremstyle{remark}
\newtheorem{rem}{Remark}

% Marks the beginning of the document
\begin{document}
\bibliographystyle{IEEEtran}
\vspace{3cm}

\title{10/A/C/5-19}
\author{EE24BTECH11040 - Mandara Hosur}
\maketitle
\newpage
\bigskip

\renewcommand{\thefigure}{\theenumi}
\renewcommand{\thetable}{\theenumi}

\section*{\textbf{C. MCQs with One Correct Answer}}

\begin{enumerate}

\item If $f(x) = \cos{(\ln{x})}$, then $$f(x)f(y)-\frac{1}{2} \left[f\left(\frac{x}{y}\right)+f(xy)\right]$$ has the value

\hfill{(1983 - 1 Mark)}

\begin{multicols}{2}
	\begin{enumerate}
		\item -1 
		\item $\frac{1}{2}$
		\item -2 
		\item none of these
	\end{enumerate}
\end{multicols}

\item The domain of definition of the function
$$y = \frac{1}{\log_{10}{(1-x)}} + \sqrt{x+2}$$ is

\hfill{(1983 - 1 Mark)}

\begin{multicols}{2}
	\begin{enumerate}
		\item (-3, -2) excluding -2.5 
		\item $\left[0, 1\right]$ excluding 0.5
		\item $\left[-2, 1\right)$ excluding 0 
		\item none of these
	\end{enumerate}
\end{multicols}

\item Which of the following functions is periodic?

\hfill{(1983 - 1 Mark)}

\begin{enumerate}
\item $f(x)=x-\left[x\right]$ where $\left[x\right]$ denotes the largest integer less than or equal to the real number $x$
\item $f(x)=\sin{\frac{1}{x}}$ for $x\neq0$, $f(0)=0$
\item $f(x)=x\cos{x}$
\item none of these
\end{enumerate}

\item Let $f(x)=\sin{x}$ and $g(x)=\ln{|x|}$. If the ranges of the composition functions $fog$ and $gof$ are $R_1$ and $R_2$ respectively, then 

\hfill{(1994 - 2 Marks)}

\begin{enumerate}
\item $R_1=\cbrak{u:-1\le u<1}$, $R_2=\cbrak{v:-\infty<v<0}$
\item $R_1=\cbrak{u:-\infty<u<0}$, $R_2=\cbrak{v:-1\le v\le0}$
\item $R_1=\cbrak{u:-1<u<1}$, $R_2=\cbrak{v:-\infty<v<0}$
\item $R_1=\cbrak{u:-1\le u\le1}$, $R_2=\cbrak{v:-\infty<v\le0}$
\end{enumerate}

\item Let $f(x)=(x+1)^{2}-1$, $x\ge-1$. Then the set $\{x:f(x)=f^{-1}(x)\}$ is

\hfill{(1995)}

\begin{enumerate}
\item $\cbrak{0, -1, \frac{-3+i\sqrt{3}}{2}, \frac{-3-i\sqrt{3}}{2}}$
\item \cbrak{0, 1, -1}
\item \cbrak{0, -1}
\item empty
\end{enumerate}

\item The function $f(x)=|px-q|+r|x|$, $x\in(-\infty,\infty)$ where $p>0$, $q>0$, $r>0$ assumes its minimum value only on one point if

\hfill{(1995)}

\begin{multicols}{2}
	\begin{enumerate}
		\item $p\neq q$
		\item $r\neq q$
		\item $r\neq p$ 
		\item $p=q=r$
	\end{enumerate}
\end{multicols}

\item Let $f(x)$ be defined for all $x>0$ and be continuous. Let $f(x)$ satisfy $f\left(\frac{x}{y}\right)=f(x)-f(y)$ for all $x$, $y$ and $f(e)=1$. Then

\hfill{(1995S)}

\begin{multicols}{2}
	\begin{enumerate}
		\item $f(x)$ is bounded 
		\item $f\left(\frac{1}{x}\right)\to0$ as $x\to0$
		\item $xf(x)\to1$ as $x\to0$ 
		\item $f(x)=\ln{x}$
	\end{enumerate}
\end{multicols}

\item If the function $f:[1,\infty)\to[1,\infty)$ is defined by $f(x)=2^{x(x-1)}$, then $f^{-1}(x)$ is

\hfill{(1999 - 2 Marks)}

\begin{multicols}{2}
	\begin{enumerate}
		\item $\left(\frac{1}{2}\right)^{x(x-1)}$ 
		\item $\frac{1}{2}\left(1+\sqrt{1+4\log_{2}{x}}\right)$
		\item $\frac{1}{2}\left(1-\sqrt{1+4\log_{2}{x}}\right)$ 
		\item not defined
	\end{enumerate}
\end{multicols}

\item Let $f:R\to R$ be any function. Define $g:R\to R$ by $g(x)=|f(x)|$ for all $x$. Then $g$ is

\hfill{(2000S)}

\begin{enumerate}
\item onto if $f$ is onto
\item one-one if $f$ is one-one
\item continuous if $f$ is continuous
\item differentiable if $f$ is differentiable
\end{enumerate}

\item The domain of definition of the function $f(x)$ given by the equation $2^{x}+2^{y}=2$ is

\hfill{(2000S)}

\begin{multicols}{2}
	\begin{enumerate}
		\item $0<x\le1$ 
		\item $0\le x\le1$
		\item $-\infty<x\le0$ 
		\item $-\infty<x<1$
	\end{enumerate}
\end{multicols}

\item Let $g(x)=1+x-[x]$ and
\begin{equation}
f(x)=
\begin{cases}
-1, & \text{$x<0$} \\
0, & \text{$x=0$.} \\
1, & \text{$x>0$}
\end{cases}
\end{equation}
Then for all $x$, $f(g(x))$ is equal to

\hfill{(2001S)}

\begin{multicols}{2}
	\begin{enumerate}
		\item $x$ 
		\item 1
		\item $f(x)$ 
		\item $g(x)$
	\end{enumerate}
\end{multicols}

\item If $f:[1,\infty)\to[2,\infty)$ is given by $f(x)=x+\frac{1}{x}$ then $f^{-1}(x)$ equals

\hfill{(2001S)}

\begin{multicols}{2}
	\begin{enumerate}
		\item $\frac{(x+\sqrt{x^{2}-4})}{2}$ 
		\item $\frac{x}{(1+x^{2})}$
		\item $\frac{(x-\sqrt{x^{2}-4})}{2}$ 
		\item $1+\sqrt{x^{2}-4}$
	\end{enumerate}
\end{multicols}

\item The domain of definition of $f(x)=\frac{\log_{2}{(x+3)}}{x^{2}+3x+2}$ is

\hfill{(2001S)}

\begin{multicols}{2}
	\begin{enumerate}
		\item $R \backslash \cbrak{-1,-2}$ 
		\item $(-2,\infty)$
		\item $R \backslash \cbrak{-1,-2,-3}$ 
		\item $(-3,\infty)\backslash\cbrak{-1,-2}$
	\end{enumerate}
\end{multicols}
                                                                                        
\item Let $E=\cbrak{1,2,3,4}$ and $F=\cbrak{1,2}$. Then the number of onto functions from E to F is

\hfill{(2001S)}

\begin{multicols}{4}
	\begin{enumerate}
		\item 14 
		\item 16 
		\item 12 
		\item 8
	\end{enumerate}
\end{multicols}

\item Let $f(x)=\frac{\alpha x}{x+1}$, $x\neq-1$. Then, for what value of $\alpha$ is $f(f(x))=x$?

\hfill{(2001S)}

\begin{multicols}{4}
	\begin{enumerate}
		\item $\sqrt{2}$ 
		\item $-\sqrt{2}$ 
		\item 1 
		\item -1
	\end{enumerate}
\end{multicols}

\end{enumerate}

\end{document}
